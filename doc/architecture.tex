\chapter{Architektūra}

\section{Idėja}

Kadangi išraiškų interpretatoriaus rašymas yra sudėtinga bei neesminė
procedūra, buvo nuspręsta pasinaudoti kuriuo nors iš jau egzistuojančių.
Taip buvo nuspręsta rašyti biblioteką Python programavimo kalbai. Kadangi
greitis yra itin svarbus, tai pati biblioteka buvo realizuota naudojant
C++ programavimo kalbą ir prijungta pasinaudojus \verb|Boost.Python| 
\cite{boost_python} biblioteką.

Tokia pasirinkta architektūra turi porą privalumų:
\begin{itemize}
  \item lengvai praplečiama – galima naudoti kartu su kitomis Python 
    bibliotekomis ir sintaksinėmis konstrukcijomis (pavyzdžiui, kartu
    \verb|xml.dom.minidom| galima braižyti itin didelio tikslumo 
    \verb|SVG| \cite{svg} formato brėžinius);
  \item platus panaudojimo spektras – galima tiesiogiai naudoti
    serveriuose, teikiant \url{http://sagenb.org} 
    arba \url{http://wolframalpha.com} tipo servisus, bei
    generuojant \LaTeX dokumentus.
\end{itemize}

Pačią biblioteką pagal atsakomybes galima suskirstyti į tris dalis:
\begin{enumerate}
  \item klasės ir funkcijos atsakingos už veiksmus su didelio tikslumo
    skaičiais;
  \item klasės atsakingos už teisingą išraiškų apdorojimą;
  \item klasės ir funkcijos atsakingos už bibliotekos Python sąsają.
\end{enumerate}
Pirmosios dvi dalys realizuotos C++ programavimo kalba, o trečioji
C++ ir Python.

\section{Veiksmai su didelio tikslumo skaičiais}

Šią dalį sudaro C++ klasė \verb|Number|, kuri realizuoja didelio tikslumo
slankaus kablelio skaičių bei tiesiogiai ją naudojančios matematinės
funkcijos: $\sin$, $\cos$, $\exp$ ir t.t. Klasė \verb|Number| yra 
apgaubiančioji klasė, kuri leidžia naudotis \verb|MPFR| \cite{mpfr} 
bibliotekos funkcijoms, objektinei paradigmai įprastu būdu.

Toliau skyreliuose aprašytos esminės idėjos, kaip buvo realizuotos kai
kurios funkcijos.

% Ji yra apgaubiančioji klasė \verb|MPFR|
% \cite{mpfr} bibliotekos funkcijoms.

% Šiai bibliotekos daliai priklauso klasė \verb|Number|\footnote{Klasė yra
% faile \verb|cpp/number.h|.} bei įvairios pagalbinės 
% funkcijos\footnote{Funkcijos yra failuose \verb|cpp/number_*.h|.}.
% Klasė \verb|Number| yra apgaubianti klasė, leidžianti naudotis 
% \verb|MPFR| \cite{mpfr} didelio tikslumo skaičių biblioteka, įprastu 
% objektinei paradigmai būdu. 

\subsection{$\pi$ reikšmės apskaičiavimas}

Yra žinoma matematinė tapatybė:

\begin{equation}
  \frac{\pi}{4} = \arctg 1.
  \label{pi_1}
\end{equation}

Taigi norėdami apskaičiuoti $\pi$ reikšmę norimu tikslumu, turime 
apskaičiuoti $\arctg 1$ reikšmę ir padauginti iš $4$. Išskleiskime
$\arctg x$ Teiloro eilute taške $0$:

\begin{align*}
  (\arctg x)'|_{x=0} &= \frac{1}{1 + x^2}|_{x=0} = 1 \\
  (\arctg x)''|_{x=0} &= -\frac{2 x}{(1 + x^2)^2}|_{x=0} = 0 \\
  (\arctg x)'''|_{x=0} &=%
    \frac{8 x^2}{(x^2 + 1)^3} - \frac{2}{(x^2 + 1)^2} = -2 \\
  \cdots\\
  (\arctg x)^{(2k)}|_{x=0} &= \cdots = 0 \\
  (\arctg x)^{(2k+1)}|_{x=0} &= \cdots = (-1)^{k+1}(2k)! 
\end{align*}

\begin{equation}
  \arctg x = x - \frac{x^3}{3} + \frac{x^5}{5} - \frac{x^7}{7}%
    + \cdots
  \label{pi_2}
\end{equation}

Gavome $\arctg x$ skleidinį laipsnine eilute. Taigi, į jį galime 
tiesiog įsistatyti $x = 1$ ir galėsime apskaičiuoti $\frac{\pi}{4}$ reikšmę
norimu tikslumu. Problema yra tame, kad kai $x$ artimas 1, tai ši
eilutė konverguoja labai lėtai, todėl reikės labai daug dėmenų, kad
pasiektume norimą tikslumą. Šią problemą galime išspręsti pasinaudoję
tokia matematine tapatybe: % TODO Įrodyti tapatybę.
\begin{equation}
  \arctg 1 = 4 \arctg \frac{1}{5} - \arctg \frac{1}{239}.
  \label{pi_3}
\end{equation}

Pasinaudoję ja galime vietoj to, kad skaičiuotume sumą lėtai 
konverguojančios eilutės, suskaičiuoti sumas dviejų žymiai greičiau
konverguojančių eilučių.
% TODO Aprėžti liekamąjį narį. Žr. alternuojančios eilutės ir
% Leibnico požymis.

\subsection{Trigonometrinių funkcijų reikšmių apskaičiavimas}

Išskleiskime $\sin x$ laipsnine eilute taške $0$:

\begin{align*}
  (\sin x)'|_{x=0} &= \cos x|_{x=0} = 1 \\
  (\sin x)''|_{x=0} &= -\sin x|_{x=0} = 0 \\
  (\sin x)''' |_{x=0} &= -\cos x|_{x=0} = -1 \\
  \cdots \\
  (\sin x)^{(2k)} &= \cdots = 0 \\
  (\sin x)^{(2k+1)} &= \cdots = (-1)^{(k)}
\end{align*}

\begin{equation}
  \sin x = x - \frac{x^3}{3!} + \frac{x^5}{5!} - \frac{x^7}{7!} + \cdots
  \label{sin_1}
\end{equation}

Gavome $\sin x$ skleidinį. Pabandykime įvertinti liekamąjį narį, jei
skaičiuosime taške $x_{0}$ ir sumuosime $n$ dėmenų:
\begin{align}
  r_{n}(x_{0}) %
  &= \frac{f^{(n+1)}(c)}{(n+1)!} x_{0}^{n+1} \\
  &= \frac{\sin^{(n+1)}(c)}{(n+1)!} x_{0}^{n+1} \\
  &\leq \frac{x_{0}^{n+1}}{(n+1)!} & \text{jei } |x_{0}| < 1 \\
  &\leq \frac{x_{0}^{n}}{n!} 
\end{align}
Taigi, kai argumentas priklauso intervalui $(1; 1)$, rezultato norimą
tikslumą gausime, kai skaičiuosime iki kol pridedamas narys pasidarys
mažesnis už norimą tikslumą. Taip pat pastebėkime, kad kuo argumentas
mažesnis, tuo seka greičiau konverguoja. Jei turime argumentą 
$x (x > 0)$ (kadangi $\sin$ yra nelyginė funkcija, tai neigiamiems 
argumentams reikės tik pakeisti rezultato ženklą), tai jį „sumažinti“ 
taip, kad jis būtų iš intervalo $[-\frac{\pi}{4}; \frac{\pi}{4}]$, galime 
tokiu būdu:
\begin{enumerate}
  \item jei $y := x - \left\lfloor \frac{x}{2\pi} \right\rfloor \cdot (2\pi)$, 
    tai $y \in [0; 2\pi]$;
  \item jei $y \in [0; \frac{\pi}{4})$, tai $\sin x = \sin y$;
  \item jei $y \in [\frac{\pi}{4}; \frac{3\pi}{4})$, tai $\sin x = \cos (y - \frac{\pi}{2})$;
  \item jei $y \in [\frac{3\pi}{4}; 1\frac{1}{4}\pi)$, tai 
    $\sin x = \sin (\pi - y)$;
  \item jei $y \in [1\frac{1}{4}\pi; 1\frac{3}{4}\pi)$, tai
    $\sin x = -\cos (y - 1\frac{1}{2}\pi)$;
  \item jei $y \in [1\frac{3}{4}\pi; 2\pi]$, tai
    $\sin x = \sin (y - 2\pi)$.
\end{enumerate}

$\cos$ reikšmę norimu tikslumu galime apskaičiuoti analogišku būdu.

\subsection{Eksponentės ir natūrinio logaritmo reikšmės apskaičiavimas}

Išskleiskime $e^{x}$ laipsnine eilute taške $0$:

\begin{align*}
  (e^{x})'|_{x=0} &= e^{x}|_{x=0} = 1 \\
  (e^{x})''|_{x=0} &= e^{x}|_{x=0} = 1 \\
  (e^{x})'''|_{x=0} &= e^{x}|_{x=0} = 1 \\
  \cdots \\
  (e^{x})^{(n)}|_{x=0} &= e^{x}|_{x=0} = 1 \\
\end{align*}

\begin{equation}
  e^{x} = 1 + \frac{x}{1!} + \frac{x^2}{2!} + \frac{x^3}{3!} +%
    \frac{x^4}{4!} + \frac{x^5}{5!} + \cdots
  \label{e_1}
\end{equation}

% TODO Aprėžti liekamąjį narį.

Dabar išskleiskime natūrinį logaritmą $\ln (1+x)$ laipsnine eilute 
taške $0$:
\begin{align*}
  (\ln (1 + x))'|_{x=0} &= \left. \frac{1}{1+x} \right|_{x=0} = 1 \\
  (\ln (1 + x))''|_{x=0} &= \left. \frac{-1!}{(1+x)^{2}} \right|_{x=0} %
    = -1! \\
  (\ln (1 + x))'''|_{x=0} &= \left. \frac{2!}{(1+x)^{3}} \right|_{x=0} %
    = 2! \\
  \cdots \\
  (\ln (1 + x))^{(n)}|_{x=0} &= %
    \left. \frac{(-1)^{n-1}(n-1)!}{(1+x)^{n}} \right|_{x=0} = %
    (-1)^{n-1}(n-1)!
\end{align*}

\begin{equation}
  \ln (1 + x) = x - \frac{x^2}{2} + \frac{x^3}{3} - \frac{x^4}{4} + \cdots
  \label{ln_1}
\end{equation}

Kadangi gautoji eilutė yra alternuojančioji ir 
$\frac{x^n}{n} > \frac{x^{n+1}}{n+1}$, kai $x \in (-1; 1)$, tai pagal 
Leibnico požymį eilutė konverguos ir paklaida bus mažesnė už pirmąjį
atmestąjį narį. Deja, kai $|x| > 1$ ši eilutė diverguoja, o mes
norime apskaičiuoti $\ln z, \forall z$, kai $z > 0$. Be to, 
kai $|x|$ artėja prie 1, tai eilutė konverguoja labai lėtai. Taigi
mums reikia kiek įmanoma sumažinti $|x|$ reikšmę. Tai galime padaryti
pasinaudodami rekursija:
\begin{enumerate}
  \item \label{ln_2} jei $z \in \left[ \frac{1}{2}; 1\frac{1}{2} \right]$, 
    tai apskaičiuojame reikšmę naudodami skleidinį, kai $x = z - 1$;
  \item jei $z < \frac{1}{2}$, tai 
    $\ln z = \ln \left( \frac{ze}{e} \right) =$
    $\ln (ze) - \ln e = \ln (ze) - 1$. Su $\ln (ze)$ vėl pradedame
    nuo \ref{ln_2} žingsnio;
  \item jei $z > 1\frac{1}{2}$, tai 
    $\ln z = \ln \left( e\frac{z}{e} \right) =$
    $\ln \left( \frac{z}{e} \right) + \ln e =$
    $\ln \left( \frac{z}{e} \right) + 1$. Su 
    $\ln \left( \frac{z}{e} \right)$ vėl pradedame nuo \ref{ln_2}
    žingsnio.
\end{enumerate}

\section{Išraiškų apdorojimas}

Šią dalį sudaro išvestinės klasės iš abstrakčios klasės \verb|Expression|.
Kiekviena iš jų reprezentuoja arba skaitinę išraišką (konstantas $e$,
$\pi$ arba sveikąjį skaičių), arba kokį nors veiksmą su kitomis išraiškomis
(sudėtį, atimtį, dalybą, $\sin$ apskaičiavimą ir panašiai). Jos yra 
atsakingos už savo reikšmės apskaičiavimą nurodytu dešimtainiu tikslumu.
Apskaičiuotas rezultatas yra grąžinamas, kaip simbolių eilutė, kurioje yra
suapvalintas dešimtainis skaičius, turintis nurodytą kiekį dešimtainių
skaitmenų.

\section{Sąsaja}

Bibliotekos Python sąsaja realizuota sukuriant apgaubiančią klasę
\verb|PythonExpression|, kurią \verb|Boost.Python| bibliotekos dėka
galima tiesiogiai pasiekti iš Python.
