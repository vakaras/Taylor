\chapter{Naudojimosi pavyzdžiai}

Pagrindinė bibliotekos klasė yra \verb|Expression|: 
\begin{lstlisting}[language=python]
  from lmath import Expression
\end{lstlisting}
Ji yra bet kokios matematinės išraiškos abstrakcija. Pati paprasčiausia
išraiška yra konstanta. Bibliotekoje realizuotos matematinės konstantos
$\pi$ ir $e$. Taip pat \verb|Expression| konstruktoriui galima paduoti
sveikus skaičius. Įvykdę:
\begin{lstlisting}[language=python]
  from lmath import Expression, pi, e
  show = lambda x: x.latex() + '&=' + x.asString(10)
  a = Expression()        # Jei nenurodyta jokia, numatytoji yra 0
  b = Expression(1)
  print r'\begin{align*}'
  print show(pi), r'\\'
  print show(e), r'\\'
  print show(a), r'\\'
  print show(b), r'\\'
  print r'\end{align*}'
\end{lstlisting}
gautume:

\begin{python}%
from lmath import Expression, pi, e
show = lambda x: x.latex() + '&=' + x.asString(10)
a = Expression()
b = Expression(1)
print r'\begin{align*}'
print show(pi), r'\\'
print show(e), r'\\'
print show(a), r'\\'
print show(b)
print r'\end{align*}'
\end{python}

Turint apsibrėžus „pamatines“ išraiškas, galima su jomis įvairiai 
manipuliuoti – sudėti, atimti, padauginti, padalinti bei pakelti
laipsniu. Pavyzdžiui, įvykdę:
\begin{lstlisting}[language=python]
  from lmath import Expression, pi, e
  show = lambda x: x.latex() + '&=' + x.asString(10)
  a = Expression(3)
  b = Expression(7)
  print r'\begin{align*}'
  print show(a + b), r'\\'
  print show(-a), r'\\'
  print show(a - b), r'\\'
  print show(a * b), r'\\'
  print show(a / b), r'\\'
  print show(pi * (a ^ 2))    
  print r'\end{align*}'
\end{lstlisting}
gausime:
\begin{python}%
from lmath import Expression, pi, e
show = lambda x: x.latex() + '&=' + x.asString(10)
a = Expression(3)
b = Expression(7)
print r'\begin{align*}'
print show(a + b), r'\\'
print show(-a), r'\\'
print show(a - b), r'\\'
print show(a * b), r'\\'
print show(a / b), r'\\'
print show(pi * (a ^ 2))
print r'\end{align*}'
\end{python}

Pastaba: numatytasis kėlimo laipsniu operatorius Python programavimo kalboje
yra „\verb|**|“, bet kadangi tokio operatoriaus nėra C++, todėl vietoj jo 
buvo perdengtas „\verb|^|“ operatorius.

Prie turimos išraiškos galima pridėti (atimti, padauginti, padalinti, 
pakelti laipsniu) ne tik kitą išraišką, bet ir sveiką skaičių:
\begin{lstlisting}[language=python]
  from lmath import Expression, pi, e
  show = lambda x: x.latex() + '&=' + x.asString(10)
  a = Expression(3)
  print r'\begin{align*}'
  print show(a + 7), r'\\'
  print show(a - 7), r'\\'
  print show(a * 7), r'\\'
  print show(a / 7), r'\\'
  print show(pi * (a ^ 2))
  print r'\end{align*}'
\end{lstlisting}
\begin{python}%
from lmath import Expression, pi, e
show = lambda x: x.latex() + '&=' + x.asString(10)
a = Expression(3)
print r'\begin{align*}'
print show(a + 7), r'\\'
print show(a - 7), r'\\'
print show(a * 7), r'\\'
print show(a / 7), r'\\'
print show(pi * (a ^ 2))
print r'\end{align*}'
\end{python}

Be šių pagrindinių veiksmų bibliotekoje, taip pat yra realizuotos
$\sin$, $\cos$, $\exp$ ir $\ln$ funkcijos, kurioms argumentas gali
būti tiek sveikas skaičius, tiek ir kita išraiška:

\begin{lstlisting}[language=python]
  from lmath import Expression, pi, e, sin, cos, exp, ln
  show = lambda x: x.latex() + '&=' + x.asString(10)
  a = Expression(3)
  print r'\begin{align*}'
  print show(sin(pi/2 + 1)), r'\\'
  print show(sin(1)), r'\\'
  print show(cos(-pi/2 + 1)), r'\\'
  print show(cos(1)), r'\\'
  print show(exp(pi)), r'\\'
  print show(exp(2)), r'\\'
  print show(ln(e^10)), r'\\'
  print show(ln(10))
  print r'\end{align*}'
\end{lstlisting}
\begin{python}%
from lmath import Expression, pi, e, sin, cos, exp, ln
show = lambda x: x.latex() + '&=' + x.asString(10)
a = Expression(3)
print r'\begin{align*}'
print show(sin(pi/2 + 1)), r'\\'
print show(sin(1)), r'\\'
print show(cos(-pi/2 + 1)), r'\\'
print show(cos(1)), r'\\'
print show(exp(pi)), r'\\'
print show(exp(2)), r'\\'
print show(ln(e^10)), r'\\'
print show(ln(10))
print r'\end{align*}'
\end{python}

% \begin{python}%
% from lmath import latex_show_as_aligned
% from lmath import Expression, e, pi
% a = Expression()
% b = Expression(1)
% print latex_show_as_aligned([
%   [e.latex(), '=', e.asString(10)],
%   [pi.latex(), '=', pi.asString(10)],
%   [a.latex(), '=', a.asString(10)],
%   [b.latex(), '=', b.asString(10)],
%   ])
% \end{python}

