\chapter{Kompiliavimas}

Biblioteka turėtų veikti tiek \verb|Windows|, tiek 
\verb|Unix| tipo operacinėse sistemose, bet dėl techninių priežasčių
ji buvo ištestuota tik \verb|Ubuntu| (tiek 64, tiek 32 bitų 
architektūros) operacinėje sistemoje. Žemiau pateikiama instrukcija
kaip paruošti biblioteką naudojimui \verb|Ubuntu| sistemoje.

\section{Kompiliavimas \emph{Ubuntu} sistemoje}

Visus bibliotekos kompiliavimui reikalingus paketus galima įdiegti įvykdžius
komandą:
\begin{lstlisting}[language=bash]
  sudo apt-get install \
    build-essential python-dev \
    libboost-dev libboost-python-dev boost-build \
    libmpfr-dev
\end{lstlisting}

Pirmiausia reikia nusikopijuoti išeities tekstus į vietą, kur naudotojas
turi rašymo teises. Tai galima padaryti arba tiesiog nukopijuojant
iš \verb|CD|, arba jei sistemoje yra įdiegta \verb|Git| versijų kontrolės
sistema \cite{git}, tai išeities tekstus galima parsisiųsti iš centrinės
saugyklos įvykdžius komandą:

\begin{lstlisting}[language=bash]
  git clone git://github.com/vakaras/Taylor.git  
\end{lstlisting}

Jei \verb|$src| yra šakninis išeities tekstų katalogas, o \verb|$dest|
katalogas, į kurį norima įdiegti Python paketą, tai tai galima 
atlikti įvykdžius tokią komandų seką:

\begin{lstlisting}[language=bash]
  cd "$src"
  chmod 755 install.sh
  ./install.sh "$dest"
\end{lstlisting}

Sėkmės atveju, į terminalo langą turėtų būti išvesta kažkas panašaus į:
\begin{verbatim}
Paskirties vieta: „${HOME}/tmp“.
Kompiliuojama.
python setup.py build
running build
running build_ext
building 'lmathcpp' extension
creating build
creating build/temp.linux-x86_64-2.6
...
Kopijuojama.
Sėkmingai įdiegta.
Patikrinti ar veikia galite tokiu būdu:
$ export PYTHONPATH="${HOME}/tmp:"
$ python
>>> from lmath import sin
>>> print sin(1).asString(50)
\end{verbatim}

Patikrinti, ar kompiliavimas pavyko galima scenarijaus nurodytu būdu:
\begin{lstlisting}[language=bash]
  export PYTHONPATH="${HOME}/tmp:"
  python
\end{lstlisting}
\begin{lstlisting}[language=python]
  from lmath import sin
  print sin(1).asString(50)
\end{lstlisting}
Turėtų išvesti:
\begin{verbatim}
0.8414709848078965066525023216302989996225630607984
\end{verbatim}

Pastaba: norint, kad biblioteka galima būtų pasinaudoti iš \LaTeX, 
kartu su paketu (aplankas \verb|lmath|) į \LaTeX dokumento išeities
tekstų pagrindinį katalogą reikia nukopijuoti ir failą 
\verb|doc/python.sty|. Tada dokumento preambulėje pridėjus
komandą \verb|\usepackage{python}|, į patį dokumentą galima bus pridėti
kreipinius į \verb|lmath| biblioteką: 
\begin{lstlisting}[language=tex]
  \begin{equation*}
    \begin{python}%
  from lmath import pi
  print pi.latex(), '=', pi.asString(32)
    \end{python}
  \end{equation*}
\end{lstlisting}

Sukompiliuotas rezultatas atrodytų šitaip:

\begin{equation*}
  \begin{python}%
from lmath import pi
print pi.latex(), '=', pi.asString(32)
  \end{python}
\end{equation*}
